\section{Introduction}

The basis of this project was the study and design of algorithms for the
partial sequencing of the human genome on a variety of sequential and
parallel architectures.  This work was done under the supervision
of Dr. Michael Minkoff of the Computing and Telecommunications Division
(CTD).  The research was done in support of a genome sequencing project
headed by Dr. Radoje Drmanac and Dr. Radomir Crkvenjakov of the Biological
and Medical Research Division (BIM).  Larry Rudsinski (CTD) also aided in
the optimization and design of the implemented code.

The algorithm consists of two main sections.  The first section is for the
generation of a sample, randomly generated, data table which will simulate
the data that will be generated in the lab by the biologists.  It will be used
for the development of the second section of the algorithm.  The second
section is the actual reconstruction of the genome from the sample data.

\subsection{Table Generation}
The generation of the table must be done so as to emulate the experimental
results.  The table is created to represent all of the data that will be
known when the attempt to reconstruct the genome is made.  For now, the
table will be created using data that would result if no errors occurred in
the experiments.  However, the experimental data will have errors;
therefore, the table generation will eventually have to be modified to
simulate false positive (true results that should be false) and false
negative (results that are incorrectly false) results.

The first step in the construction of the table is to generate a
random sequence that shall serve as the genome.  This sequence consists of
random combination of the letters 'A', 'T', 'C', and 'G'.  These letters
represent the four nucleotides that are found in DNA sequences.  These
nucleotides are known as : adenosine (A), thymidine (T), cytidine (C), and
guanosine (G).  This sequence must be constructed in a truly random way
for all later statistical studies to be considered true.  The length of the
sequence shall be given in base pairs (bp), where base pairs is the number
of the characters in the sequence.  All sections of the sequence
shall also have their lengths given in this
manner.  This sequence shall be referred to as the 'genome' or plainly
as the 'sequence'.

The next step in the generation process is to break the sequence up into
smaller sections
called 'fragments'.  These fragments are to have a random amount of overlap
with the previous fragment, however there will be set values for the
maximum and minimum amount of overlap two adjacent fragments can have.

Third, a table of 'probes' should be generated.  The probes are
combinations of nucleotides of varying length.  At present all combinations
of length six are being considered.  The resulting table of probes looks like
:
\begin{center}
\begin{tabular}{c}
AAAAAA \\ 
AAAAAT \\ 
AAAAAC \\ 
AAAAAG \\ 
AAAATA \\ 
 \vdots \\ 
GGGGGG
\end{tabular}
\end{center}

Once the probes and fragments have been generated, the table can be
constructed.  The table is to show if a given probe is in a given fragment.
The process used to determine if a probe is contained in a fragment, or how
this data is stored is of little consequence at this juncture, but it
will play an increasingly greater role as the study continues.

\subsection{Reconstruction}
The reconstruction portion of the algorithm is to actually rebuild the
sequence (genome) from the table that is given.  In the test cases, it
would be an attempt to reconstruct the generated sequence in the first step
using the table that was created later in the same step.  In the actual
problem, complete reconstruction is not possible due to areas in the genome
that are repetitious.  The reason that repetition cannot be treated by the
algorithm shall be explained later.

The general idea behind the reconstruction is that given a specific probe
from a given fragment, one can look in the table to find a probe that is
also in the fragment that overlaps the current probe with a maximum amount
of overlap.  This new probe shall be called the 'extension' probe.  In the 
test case, all probes, except for the final probe in a fragment, will 
have at least one probe
(with a maximum of four) that have this amount of overlap.  In the real
case, it is possible for a probe, other than the final probe in a fragment,
to not have an extension due to a false negative result in the experiment.

If a probe has more than one extension, then a branching point has occurred.
This is called a branching point because two distinct paths are possible.
This branch can be resolved in more than one way.  Two possibilities are
to look in adjacent (overlapping) fragments, or to look in similar 
sequences.  The best way to resolve this has not yet been determined, but
the process that will be implemented will be a combination of different
methods.

A problem arises when a section of a fragment repeats, or if a probe
overlaps itself (such as AAAAAA).  This problem arises because there is no
way to determine how many times these sections should be repeated.  The
best way to handle this situation is to note it and move on, because the
repetition will eventually end.
Also, the biologists are not very interested in these
sections of repetition.

Since the proper method for reconstruction has not yet been decided on,
most of the research has been on generating data that will help this
decision.  Also, a lot of attention has been given to deciding which
architectures are best suited for this problem.
