\section{Statistical Considerations}

The first step in designing an algorithm for the reconstruction of the
genome is to find the scale of the problem.  The method under
consideration for the reconstruction is to extend from
probes to a certain length (20-30bp) and then find a method to determine
how to combine these extensions to reconstruct the fragments and the sequence.
A major concern in this approach is the number of possibilities that might
occur when
extending from a probe to a specific length.  It is obvious that as the
extension length increases, the possibilities increase.  Therefore, there
is a need to balance the length of the extensions and the number of
extensions that result.

A code was written that would accept a table of data as
input and return the number of possible extensions that occur for
various extension lengths.  This Code extends on every probe that
occurs in each of the fragments.  It keeps a total for the number of
possible extensions that occur in the fragment for each of the extension
lengths.  It also
notes the number of probes that are in the fragments.  Knowing the number
of probes that are in the fragment allows for the calculation of the
average number of extensions for a probe in the fragment by dividing the
totals for the fragment by the number of probes that are present in the
fragment.

In this code, the probes are referred to by their position in the data table.
The following table shows the correspondence between the probes and the
numbers that refer to them :
\begin{center}
\begin{tabular}{||c|r|c||} \hline
Probe  & Reference \# & Bit Representation \\ \hline
AAAAAA & 0 & 000000000000 \\
AAAAAT & 1 & 000000000001 \\
AAAAAC & 2 & 000000000010 \\
AAAAAG & 3 & 000000000011 \\
AAAATA & 4 & 000000000100 \\
\vdots & \vdots & \vdots \\
GGGGGA & 4092 & 111111111100 \\
GGGGGT & 4093 & 111111111101 \\
GGGGGC & 4094 & 111111111110 \\
GGGGGG & 4095 & 111111111111 \\ \hline
\end{tabular}
\end{center}
Each of these numbers is represented in memory in the same fashion as the
bit representations used in version 6 of the data table construction code.
Representing the probes this way allows for probes of all lengths to be
referenced without any changes to the code.

It is interesting to note that the four possible extensions to a probe are
next to each other in the table.  Knowing this, if the first extension is
calculated, the other three are exactly determined.  Considering the bit
representations for these numbers, the extension can be obtained by zeroing
the left two bits in the number and then left shift the number two bits
with a zero fill.  For example, 'CGGGGG' has 'GGGGGA', 'GGGGGT', 'GGGGGC'
and 'GGGGGG' as its extensions.  It can be seen in the table that all four
of these can be determined by knowing that 'GGGGGA' is the first of these,
and that the other three follow in the table.

The bit representation for 'CGGGGG' is '101111111111'.  In order to
discover that 'GGGGGA' is the first of the extension probes, the left two
bits from the bit representation of 'CGGGGG' must be removed to obtain
'1111111111'.  Next, the new bit pattern must be shifted left two bits with
a zero fill.  The result of this operation is the pattern '111111111100'.
This new pattern is the pattern for the probe 'GGGGGA', which is what was
needed to be obtained.  The extension probes for any probe can be
determined in this fashion.  It is also not dependent on the probe length.

Table~\ref{'table2'} is a sample of output from the code.  The data was generated from a
table that was created using fragment lengths of 500bp and probe length of
6bp.

\input table2

The routines that generate the data table should be modified so the
fragments do not have overlap regions when generating the data for
statistical analysis.  The reason for this is that regions of overlap will
have a greater influence on the results of the analysis.  The results will
not be statistically true if there are regions of overlap.  When doing an
analysis to find the average number of extensions from a given probe, the
same number of probes should be used from each fragment.  If one fragment
contributes data from more probes than the others, it will have a greater
influence on the outcome and the results will not be true again.

Table~\ref{'table3'} was generated from 100 fragments, each contributing
data from 3 probes that were chosen randomly from the possible probes in
the fragment.  The data in the table is the average number of extensions
that each probe had when extending to the given lengths.  Extension lengths
include the length of the initial probe which was 6bp.

\input table3

In trying to determine an overlap between two fragments using only the data
table, all probes that are in both of the fragments must be considered.
However, if a probe is in both of the fragments, but outside the region of
overlap, there is no way of determining that it is not suppose to be in the
overlap region.  It is interesting to compare the possible extensions
from a region of overlap to the possible extension from
non-overlapping fragments of the same length as the overlap region.  It is
obvious that the number of possible extensions from an overlap region should
be slightly higher
because probes could be considered that are in both fragments, but not in
the overlap.  This increase
is of particular interest.

Table~\ref{'table4'} shows the difference
between the number of possible extensions
from an overlap region and a non-overlapping fragment of the same
length.

\input table4
