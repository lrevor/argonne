\section{Optimization and Analysis}

During the creation of the code for counting expansion possibilities, an
effort was made to optimize the code for use on various machines and
architectures.  It was decided to analyze the performance on the various
machines because the reconstruction code will behave in a fashion similar
to that of the expansion code.  Because of the size of the genome problem,
it will be necessary to know which architectures are best suited for this
type of problem.  The computers that were used in this analysis are :
\ariel, \achilles, \hp, \gamma, \alxmp\ and the \delta.  A version of the 
code was also
written implementing P4
\footnote{P4 is based on "Argonne Macros" which is
documented in the book "Portable
Programs for Parallel Processors", by Boyle, Butler, Disz, Glickfeld, Lusk,
Overbeek, Patterson, and Stevens.  Major contributions to P4 were also made
by Bob Olson, University of Illinois, Remy Evard, University of Oregon, and
David Leibfritz, Argonne National Laboratory.}
to simulate an MIMD
distributed memory system across
a network of workstations.

The first implementations of the code were run on workstations because of
the ease in code development and accessibility.  The timings involved the
expansion out to 30bp from every probe in twenty fragments.  Each of the
computers expanded on the same table, so the amount of work being done on
each of the machines is identical.  Table~\ref{'worksta1'} shows the results
of the timings on the different machines.

\input worksta1

The next implementation was on a single processor \alxmp.  Transporting the
code and compilation required little to no effort from the workstation
versions.  A second version was written in Cray assembly language (CAL) to
utilize the Cray's vector registers to minimize the accesses to main
storage.  Only the routine that actually counts the number of expansion
possibilities was rewritten in CAL.  The rest of the code was left as is.
The timings for the \xmp\ is given in Table~\ref{'cray1'}

\input cray1

The timings for the Cray using the original routines show that the Cray's
extra architectural enhancements do not apply to this type of problem.
It is interesting to note that the \achilles\ and \hp\ both had better
timings than the Cray until the CAL routine was written.  Due to the
expense of using the Cray, it is not a very cost-effective solution to the
problem.

Since each of the fragments are totally independent from each other, the
expansions for each of the fragments can be computed simultaneously.  Because
of this, the code was ported to an \gamma (Gamma).  Since the Gamma is an MIMD
distributed memory system, the code had to be split into two sections.  The
main (driver) section of the code handled the generation of the table data
and the control of the message passing to the independent processors.  The
node section of the code was to count the extension possibilities for a
given fragment.  Because the system is distributed memory, each of the
processors only knew the data for the particular fragment it was working
on.  This data is passed to the processor by the driver section when the
node becomes available.

The revision to the code required a lot of changes to the structure of the
code, but the portions of the code that did the actual computations remained
the same.  The code was set up for the driver section to run on the front-end
(\intel) of the Gamma, while the node section was put on each of the
processors.  Table~\ref{'gamma1'} shows the timings for the Gamma using 1, 2,
4, and all 8 of its hypercube processors.  Note that the timings for the
Gamma are the maximum time needed by one of its processors.  Also, since
each of the Gamma processors is dedicated, it keeps track of real time from
when the node process is started.  Unlike the other machines, time spent
waiting to receive messages is included.

\input gamma1

This table shows that the Gamma was not as efficient as some of the workstations
when using only one node.  When the number of nodes is increased by one,
the comparison between the Gamma and the workstations becomes more
interesting.  As the length of the fragments increase, the Gamma
increasingly outperforms the workstations; however, at the smallest
fragment length, the Gamma did not do as well.

In preparation for running the code on the \delta (which is possibly the
world's fastest supercomputer), the code that was used on the Gamma was
modified so the driver section of the code would run on one of the Gamma's
processors as well.  This version takes full advantage of the Gamma's
hypercube message passing system.  The decrease in time required for
message passing is shown by the timings.  The timings for this version is 
given in Table~\ref{'gamma2'}.  Because the
driver section requires a node in this version, the number of processors
that can be used by the node section is decreased by one.

\input gamma2

This new code was then ported to the \delta\ (Delta).  Due to the number of
processors on the Delta (512),
the size of the problem used in all of the prior timings is trivial.
Because of this the number of fragments was increase to 512 (from 20) so 
the Delta's
timings would be of interest.  Table~\ref{'delta1'} shows the timings that
were done on the Delta.  

\input delta1

The Delta uses a Mesh message passing system
unlike the Gamma's Hypercube.  The Mesh size refers to the size of the grid
that is being used by the program.  To find the actual number of nodes
being used, all one has to do is multiply the two coordinates.

After seeing the these results, we thought that it would be interesting to
see how a network of Sun workstations would compare.
Therefore, the original Gamma code was converted to P4 so that it could be
run parallel on a network of workstations.  The conversion from \gamma\
message passing calls to P4 message passing calls required very little
work.  However, when using P4, all node processes must be self terminating.
This requirement did not exist on the Gamma.  This required a little change
in the structure of the code that was used on the nodes.

When running the P4 version, a \achilles\ was used to run the driver section
of the code, while the nodes processes were all run on a \ariel.  
Table~\ref{'p41'} shows the results of these timings.  The time shown
is the minimum node time.  Unlike the Gamma this time is user time and not
real time.  This means that time waiting for messages is not included in
the time.

\input p41

After seeing the results of these timings, it is clear that this type of
problem is best solved in parallel.  Out of all the versions implemented, I
believe the most cost effective version is the P4 version running on a
network of workstation.  It is not as fast as the Gamma, but it is not far
behind.  However, the Gamma version can be easily scaled up to run on more
processors by porting it to the \delta, while the P4 version requires the
addition of more workstations.
