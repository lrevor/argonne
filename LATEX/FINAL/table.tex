\section{Table Generation}
In writing the code for the generation of the table of data, the
two main computer science issues had to be taken into consideration.  These
two issues are speed of the algorithm and storage efficiency.  As a result,
six versions of the algorithm were written and implemented.  Each was written
to either improve the speed or storage efficiency.  A
description of each of the different versions and their advantages is given
below.

Table~\ref{'table1'} shows the timing for each of the six
routines.  Timings were done at differing fragment lengths, but the number
of fragments was held constant at 100.  The probe length that was used was
six.  Since the number of probes is given by the equation $4^p$ where $p$
is the length of the probes, there were a total of 4096 probes.  All of the
timings were done on a \achilles.  The data is given in seconds.
\input table1

\subsection{Version 1}
Since this version is the first version that was implemented, it is
very straightforward in its logic and data structures.  The sequence, when
generated, is stored as a character string, where each element is one of the
letters 'A', 'T', 'C', or 'G'.  After the sequence is generated, the position
in the sequence where each of the fragments are to start is calculated. 
The fragments are then copied out of the sequence and stored into an array
of character strings where each element of the array is a distinct
fragment.  The probes are generated as a character string and stored to a
file.

The table, in this version, is a two-dimensional array of logical data
type.  This is the obvious choice, because all that needs to be stored
is a true or false value.  The table is generated by reading a probe
from the file, then using a string comparison to see if the probe is
contained in each of the fragments.  If a fragment contains the probe, then
a true value is placed in the table at the appropriate position.
If the probe is not present in the fragment, then a false value is
entered.  This process is repeated for all of the probes.

The basic memory requirement (in bytes) for data of this version is given 
by :
\[ s + m*n + m*4^p \]
where :
\begin{center}
\begin{tabular}{ccl}
$s$ & = & sequence length \\
$m$ & = & number of fragments \\
$n$ & = & fragment length \\
$p$ & = & probe length
\end{tabular}
\end{center}

\subsection{Version 2}
In Version 2, the table is created in a different manner.  In this version,
each of the fragments are scanned to see which probes are contained in it.
Each probe-length section of the
fragment is copied and converted to a number that represents the probe.
Using this number, the position in the table that corresponds to the
fragment and the probe is marked true.  Since the true values will be
marked in this fashion, the table must be set to all false values before
the routine begins.

The method for converting a probe to a number involves considering the
probe to be a base-4 number, because each position in the probe has four
distinct values.  Letting the letters 'A', 'T', 'C', and 'G'
represent the numbers 0, 1, 2, and 3 respectively, a conversion can be
easily obtained.  Therefore, the probes are now represented by the numbers
0\ldots $4^p-1$.  Next, the base-4 number must be converted to a base-10
number for use by the program.

Since pattern matching is no longer used, the probes do not need to be
generated.  As a result of this, disk space is no longer needed to
store the file of probes.  However, there is no saving in the amount of
memory needed by the program for the data.

\subsection{Version 3}

In this version the fragments are no longer stored separately from the
sequence.  A vector is now used to point to the first element in the
sequence that is contained in the fragment.  Since all of the fragments are
of the same length (in the test case), each of the fragments can be exactly
determined by looking at the fragment length section of the sequence
starting at the character pointed to by the vector.

Since the probes are no longer stored separately, there is a decrease in
the amount of storage required, although allocation for the vector is
now necessary.  The new storage requirement is :
\[ s + m*w + m*4^p \]
where :
\begin{center}
\begin{tabular}{ccl}
$w$ & = & word length \\
\end{tabular}
\end{center}
Therefore, there is a savings in the storage requirement when $w<n$.

\subsection{Version 4}
This version of the algorithm was written primarily for comparison.  It is
the same as Version 1 with two exceptions.  The probes are stored in memory
and not in a disk file.  The other difference is that the fragments are
not stored separately, but are referenced as in Version 3 above.  The
table is constructed as it was in Version 1.

With this version I wanted to show the difference in time for the pattern
matching routine and the conversion routine.  As I predicted, the
conversion routine was less time consuming.

The storage requirement for this version is :
\[ s + p*4^p + m*w + m*4^p \]

\subsection{Version 5}
Version 5 was written due to the need to reduce storage requirements.
It is an improvement to Version 3 above.
This version stores the table using one bit for each element.  Two routines
were written in C to allow for the setting and reading of bits in the table.
Because of the time needed to set up a subroutine call, it was decided to
set the bits for a fragment at a time.  However, setting the bits in this
fashion causes the need for another vector to store the information on what
bits in the fragment are to be set.

The storage requirement for the table is now decreased by a factor of 8.
However, for
smaller sized tables, the savings is nullified by the need for the vector.
The requirement for this version is :
\[ s + m*w + 4^p*w + m*\frac{4^p}{8} \]
Which is less than Version 3 when $w<\frac{7m}{8}$.

As it is noted in the table above, this version of the routine provided the
fastest time out of all the six versions.  The time is a noticeable decrease
when compared to the previous four versions.

\subsection{Version 6}

This version is a continuation of the improvements made in Version 5.  It
was written to make the code portable to a wider variety of machines.  In
the previous version, the sequence prevents compilation on some
machines, because these machines do not allow character strings longer than
a specific preset length.
The sequence in this version is stored in bits.  Two bits are allocated
for every character that was needed in the previous versions.  In the
sequence, the bit patterns 00, 01, 10, and 11 represent the letters 'A',
'T', 'C', and 'G' respectively.

Since the sequence is in a different format, the scanning of the fragments
is changed to accommodate the differences.  Instead of
copying $p$ characters out of the fragments when they are scanned, $2*p$
bits will be copied out into an integer variable.  When it is copied out of
the fragment, the conversion in the previous version is
no longer needed.  The reason for this is because the
conversion is automatically done by the hardware when the value is
referenced.

The storage requirement for this version is :
\[ \frac{s}{4} + m*w + 4^p*w + m*\frac{4^p}{8} \]
