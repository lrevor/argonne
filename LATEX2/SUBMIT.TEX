\section{Submitting a DQS Job}

Since the major task for most users will be submitting jobs, this
topic will be covered first.  As mentioned earlier, jobs are submitted
to the DQS system by submitting a unix shell script.  Therefore, most
of this section will deal with describing the shell script and how to
customize it for DQS.

A shell script is a standard unix text file.  It can be created by
using any text editor (vi, ed, emacs, xedit, etc.).
A DQS job shell script has two distinct purposes.  The first is define the
environment needed by the job.  The second is to give the commands that
comprise the job itself.
Listed below is a sample script that could be submitted to the
DQS system to compile a sample FORTRAN code and then execute the
compiled program.

\begin{verbatim}
# This is a sample script file for compiling and
#  running a sample FORTRAN program under DQS.

# We want DQS to send mail when the job begins
#  and when it ends.

#$ -mu EmailAddress
#$ -mb
#$ -me

# We want to name the file for the standard output
#  and standard error.

#$ -eo test.out

# Change to the directory where the files are located.

cd TEST

# Now we need to compile the program 'test.f' and
#  name the executable 'test'.

f77 test.f -o test

# Once it is compiled, we can run the program.

test

# End of script file.
\end{verbatim}

This script file is a short example, but it shows that script files
are easy to use.

In this script file, there are three types of lines.  These are
comment lines, DQS command lines, and shell command lines.  Each of
these types will be described below.

Any line in the script file that has the '\#' character in column one and is
not followed by a '\$' or a '!' is a comment line and is not considered
by DQS or the unix shell.  They are included in the file to
provide information on what the file is supposed to be doing.  Included
in this script file are several comment lines to help explain what it
is doing to those who are new to script files.

Lines in the script file that begin with \#\$ in columns 1 and 2 are
commands to DQS.  Our example contains four such commands.  The
{\bf -mu} command
tells the \qmaster who to send all mail to.  The {\bf -mb} command
signifies that
mail should be sent when the job begins, while {\bf -me} states that
mail should
be sent when the job completes.  The {\bf -eo} command specifies the
file that all
output to {\bf stdout} and {\bf stderr} (standard output and standard
error) should be redirected to.
All DQS commands are read when the job is
submitted regardless of their position in the script file.
These same DQS commands could have been specified on the {\bf qsub} command
line when the job was submitted.
Any valid command line
parameter for the \qsub command can be placed in the scriptfile as a
DQS command.  Users should be aware of the parameters that are present in
the scriptfile so they do not unintentionally replicate the parameter.
If more than one resource request is present, DQS will assume that the job
will need the stated resources and reserve them for the job.  Parameters
other than resource requests should not be duplicated.
These parameters will be
discussed later along with the \qsub command and are given in appendix A.

The final type of command in a DQS script file is the unix shell
command.  Any line in the script file that does not begin with the
'\#' character (with the exception of the '\#!' combination)
is considered to be a unix shell command.  When the job
is run, each shell command will be executed in order starting from the
top of the file.  These commands will behave in the same manner as if
they were issued from the unix prompt.  For testing, a DQS script file
can be run as a standard script file from the unix prompt if it does not
contain any references to environment variables that are set by DQS.
When this
is done, all DQS commands will be treated as comments, because
they all begin with the '\#' character.  To execute a DQS script file
, the user must only give it execute privileges with
the unix {\bf chmod} command.

Once the script file is written, it can be submitted to DQS.
The routine that submits jobs to DQS
is called {\bf qsub}.  Two examples of \qsub calls to submit the above
script file are :
\begin{verbatim}
qsub -G groupname 1 sub
qsub -q queuename sub
\end{verbatim}
where groupname is the DQS \group that is being requested, queuename is
the DQS \queue that is being requested, and sub is the name of the
scriptfile described above.

In the above examples, the -q and -G parameters are both resource
requests.  It should be noted that both of the \qsub commands have a
resource request.  A job will not be accepted without specifying a resource
on the command line or in the job script.  If a resource is not specified,
DQS would not
know where to send the job.
The '1' after groupname in the first
example tells DQS that you will only need one \queue
in the \group that is specified.  This number should only be
greater than one for a parallel job under \pvm.  If the job is not a
parallel job, more than one resource should not be requested and the
number after the queuename should be left at '1'.  In any case, a value
must be specified for the job to be accepted.

In the first \qsub example, the job 'sub' will be sent to one of the
\queues in the \group 'groupname' and then executed.  In the second \qsub
example, the job will be sent to the \queue 'queuename'.

The syntax for the \qsub command is :
\begin{verbatim}
qsub [queue] [parameters] scriptfile
\end{verbatim}
There are several command line
parameters that can be given with the \qsub routine.  A list of these
parameters is given in appendix A.
These options may be used either on the command line for \qsub, or
placed in the job script file.  They will have the same effect in either
place.
