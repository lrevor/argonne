\section{Monitoring Jobs}

Once a job has been submitted to the DQS system, it becomes almost
invisible to the user.  Because of this, DQS has routines included
with it that allow a user to keep up with his job.

The first method is to have the DQS system send mail messages about
the job.  DQS can be told to mail a user when the job is started,
finished, suspended, or enabled.  All of this is done using the \qsub
command which is described above.  First, the user that will receive
the mail must be specified.  This is done with the -mu option of
\qsub.  After this is set, any of the -mb, -me, or -ms option can be
specified which tell DQS to mail at the beginning of a job, end of a
job, or when a job is suspended/enabled respectively.

The second method is to query DQS about the status of the job.  This
is done with the {\bf qstat} command.  The \qstat command reports
information on each of the \queues and all of the jobs that are pending.
For every \queue, it gives
information on the \queue and also on the the job (if any) that is
running on the \queue.
Users can use this information to see the
status of his jobs.

The \qstat command gives a lot of useful information, but some of the
information is often not very relevant.  Because of this, the \qstat command
has several command line options that allow the user to tell DQS what
information that he wants reported.  These options are given in
appendix A.
