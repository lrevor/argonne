\section{Introduction}

DQS\footnote{Developed at the Supercomputer Computations Research
Institute at Florida State University by Tom Green and Jeff Snyder} -
Distributed Queuing System - is an effective method for
distributing the batch workload among multiple unix-based machines.  In
doing so, it increases the productivity of all of the machines and
also increases the number of jobs that can be completed in a given
time period.  Also, by increasing the productivity of the
workstations, the need for outside computational resources is reduced.

A job is submitted to the DQS system via a job shell script.
The job script is a standard unix shell script (Bourn shell, C shell, or
Korn shell) that will executed on one of the DQS machines when the
appropriate resources become available.
The script can also be used to tell DQS to do the
following :
\begin{itemize}
\item Initiate the script under a specific unix shell.
\item Write all output sent to {\bf stdout} and {\bf stderr} to a
specified file.
\item Mail a specific user when any changes are made to the status of
the job.
\item Prepare the system for running a parallel job under
\pvm\footnote{\pvm was developed at the University of Tennessee, Oak
Ridge National Laboratory, and Emory University} - Parallel Virtual
Machine.
\item Specify the type of machine you are requesting that the
job should be run on.
\end{itemize}
The script file can be used to do multiple tasks.  By
combining shell commands with DQS commands, the user will be able to
use the system more efficiently.  A proper understanding of the script
file is the key to learning how to run existing code under the DQS
system.

Once the job script has been submitted to DQS, it is scheduled
and then sent to one of the machines on the system when one
becomes available.  The process DQS uses
to distribute the workload and decide which machine each job will be
sent to will be
discussed later.

Included in the system are routines that were developed
for the user to have control over his jobs.  These routines provide the
ability to check the status of a job(s), delete a job from the queuing
system, and check on the availability of resources.

For increased ease of use, the DQS system provides an X windows
interface.  All users of the system that are running under X windows
can use these routines.  With these routines, the user can
accomplish any task that he could have done from the command prompt
with a little more ease.  Also, the X interface allows users to get an
interactive xterm window through DQS if he feels the job must be run
interactively.

The DQS system was also developed with console users in mind.
Routines are provided that monitor the X window server for activity to
determine if the interactive user is at the terminal.  If activity is
detected, then DQS can suspend the queues that are on the
machine.  If the queues are suspended and the terminal has been inactive for
a given period, DQS will unsuspend (enable) the queues on the machine.

Overall, DQS provides efficiency, greater productivity, and increased
cost efficiency while still maintaining functionality.
