\section{Appendix A}

The following is a quick reference for the syntax of the commands covered
in this document.

\subsection{qsub}

{\bf qsub} allows a user to submit a batch job to DQS.

{\bf SYNTAX}
\begin{verbatim}
qsub [queue] [parameters] scriptfile
\end{verbatim}

{\bf PARAMETERS}
\begin{description}
\item[-e fname] Tells DQS to redirect all output to STDERR to
the file specified by fname.  DQS defaults to jobnumber.jobname.ERR if
this option is not specified.
\item[-o fname] Tells DQS to redirect all output to STDOUT to
the file specified by fname.  DQS defaults to jobnumber.jobname.OUT if this
option is not specified.
\item[-eo fname] Tells DQS to redirect all output to STDERR and
STDOUT to the file specified by fname.
\item[-cwd] The cwd flag tells DQS to execute the script from the same
absolute directory that the job was submitted from.  If this is not
specified, DQS will default to the users home directory, and the files
will not be found, unless they are in the home directory or the script
file changes the directory via the cd command.  If the absolute pathnames
are not the same across the machines, then the user should use the cd
command instead of setting the -cwd flag.  If the user sets the cwd flag
and the machine cannot find the directory, then it will default to the
users home directory on that machine.
\item[-mu EmailAddress] Tells DQS who to send mail to if the -mb,
-me, or -ms flags are set.  This must be set if any of the mail flags
are set.
\item[-mb] Tells DQS to send mail to notify when the job has begun.
\item[-me] Tells DQS to send mail to notify when the job has finished.
\item[-ms] Tells DQS to send mail to notify when the job has been suspended
or enabled after a suspend.
\item[-G groupname num] This option tells DQS that the job
will need 'num' \queues in the \group 'groupname'.  More than one -G
parameter may be specified, and they can be used in conjunction with
the -q parameter.  However, only one resource should be specified
unless the job is a parallel job under \pvm.  If the job is not a
parallel job, it will be keeping resources unavailable that could be
used by other jobs.
\item[-q queuename] This option tells DQS that the job will need to use the
\queue 'queuename'.  More than one -q parameter may be specified, and
they can be used in conjunction with the -G option.  However, users
should not specify a specific \queue unless the code was written for a
specific machine.
\item[-pvm] This flag tells DQS that the job will be using \pvm.  DQS will
then set up the environment appropriate for running a \pvm job by
creating the \pvm host file and starting the \pvm daemons on each of the
allocated resources.  This process is invisible to the DQS user.
\item[-M queuename] The tells DQS that the \pvm master should be on the
\queue 'queuename'.
\item[-s shell] This specifies the shell that the script needs to execute
under.
\item[-v] This is the verify option.  With this option set, the command
line and script file are parsed and the requests are then written to
STDOUT.
\item[-H] With this flag set, all request are considered to be hard.  This
means that the job cannot be executed until all resource requests are
available.  This is the default for all jobs.
\item[-S] This sets all of the resource requests to be considered soft.
The job can be executed even when all of the resources are not
available.
\item[-r] This enables the job to be restarted by DQS if a system error
occurs while it is running.  This is the default.
\item[-nr] This tells DQS not to restart the job if a system error occurs
when
the job is running.
\end{description}

\subsection{qstat}

{\bf qstat} allows users to see the status of the \queues and jobs in the
DQS system.

{\bf SYNTAX}
\begin{verbatim}
qstat [parameters]
\end{verbatim}

{\bf PARAMETERS}
\begin{description}

\item[-s]  Reports only the \queue name, architecture, and the status
of the \queue.  It also reports the job id and job owner if there is a
job running on the \queue.

\item[-A arc]  Reports information only on \queues of the given
architecture type.

\item[-G group]  Reports information only on the \queues that are in
the \group specified.

\item[-q queue] Reports information only on the specified \queue.

\item[-r] Sort the \queues in increasing order of load average.

\item[-u username] Report only information on jobs owned by the user
specified.

\end{description}

\subsection{qdel}

{\bf qdel} deletes a job from the DQS system.

{\bf SYNTAX}

\begin{verbatim}
qdel jobnumber
\end{verbatim}
