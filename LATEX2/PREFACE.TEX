\section {Preface}

With the continuing decline in cost and size of high powered computing
resources, scientists and scholars are finding that the workstation they
have sitting on their desk for word processing could be used to do some of
their scientific computing instead of paying large amounts of money for
outside resources.  The question then becomes how to get the most out of
ones workstation.  The answer to this question is a queuing system.
However, while a queuing system will take full advantage of the resources
on one machine, CPU cycles on other workstations in the same company are
remaining untapped.  Distributive queuing system are now being looked into
to take advantage of all of the workstation on a given network.

While I was at Argonne, my project was to do research into distributed
queuing system.  The system I spent the majority of my time on was the
Distributed Queuing System (DQS) which is currently being developed at
Florida State University by Tom Green and Jeff Snyder.

My first task when I arrived was to obtain the system and implement it for
testing and debugging.  I found many errors in the first version I
obtained, but the system was still very usable.  I spent a good deal of
time looking into the source, fixing problems and reporting these problems
to the people at Florida State.  During the Summer, we were able to improve
the quality of the system greatly.  

Another of my tasks was to introduce the system to various people in
differing divisions around the lab.  Also, I was able to teach a class
offered by CTD on the subject of Distributed Queuing systems.

The rest of this paper is the user guide to DQS that I wrote because of the
lack of documentation for the system.  A DQS administrators guide was also
to be written, but time fell short.
