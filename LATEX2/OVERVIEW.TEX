\section{Overview}

The DQS system has one routine that controls the system.  This routine
and the machine that it runs on are called the {\bf qmaster}.  The
\qmaster is responsible for processing new requests (jobs submissions,
queries, etc.), keeping track of already existing jobs, monitoring the
status of all of the machines known to DQS, and updating all of the
files used for accounting and management of the system.

All requests that are submitted to the DQS system are processed by the
\qmaster.  This does not mean that the user has to be logged on to the
\qmaster in order to submitt a request.  A request can be submitted
from any machine the DQS system considers to be a trusted host.  When
the request is submitted, it is passed across the TCP/IP network to
the \qmaster for processing.

When a user submits a job, he must specify the resources that he wants
the job to use.  This is to tell DQS how many and what kind of
machines he needs for the job.  The two different types of resources
in the DQS system are the {\bf queue} and the {\bf group}.

The \queue is the standard computer science first-in-first-out (fifo)
queue.  Each \queue resides on a particular machine.  A job is sent
to a machine by submitting it to the \queue that resides on the
machine.  If the \queue is busy with a job when a new job is submitted,
the new job simply waits its turn in line in the \queue.  More than one
\queue may reside on a particular machine, but this is not recommended
except for the purpose of having \queues of differing priorities or if the
machine has sufficient resources for handling more than one \queue.

A \group is a collection of \queues, where each \queue inside the \group
resides on a machine of a particular architecture.  This is how DQS
solves the problem of having multiple types of architectures.  When
the user specifies which \group he wants, he is in effect specifying
the class of machines he would like.  When a job is submitted to a
\group, DQS sends the job to a \queue in the \group that is available.
If all of the \queues are busy when a job is submitted to the \group,
the job will wait its turn in line at the \group level.  This feature of
waiting at the \group level is what allows for load leveling across a
network of machines.

It is better for all users to submit jobs to a \group rather than a
\queue due to the load leveling capabilities of DQS.  If all jobs are
submitted to the \group, the amount of waiting time will be minimized,
because all of the machines inside the \group will be efficiently
utilized.  The only reason a user should submit a job to a
particular \queue is if the code was designed to run on a particular
machine instead of a class of machines.

